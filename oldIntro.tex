Individuals are unique in the way they speak, look, and even walk {\color{red}[ref?]}. This has opened the door for various research fields such as face and voice recognition systems. Similarly, we are attempting in this paper to examine if individuals and specifically artists are also unique in the way they sketch and if they are, then to what extent can this uniqueness be used to identify and classify their sketches or to detect fraudulent sketches?

To our best knowledge there does not exist a similar work such that various sketches of different artists are being analyzed in order to determine their authorships based on extracted local features that are able to convey information related a to a particular artist's style as the work we provide in (SAR). {\color{red} [This is a run on sentence than needs to broken up and made simpler.]} This in turn is expected to serve in a variety of applications, such as fraud detection in design patents to be used for litigation, brand marking, style training and reproduction, handwriting verification and artistic style analysis. {\color{red} [This needs more explanation. Instead of a list, elaborate more on the applications because they are the main justification for our work.]}

Our proposed (SAR) method consists of: 1. Obtaining sketches drawn by different artists 2. Extracting the boundary and the internal curves from a sketch; 3. Cutting the curves into unequal length segments that can expose its authorship; 4. Examining the characteristics of those segments and representing each sketch by features derived from these characteristics, then; 5. Discriminating and classifying authorship based on their new representation as shown in our development pipeline ~\ref{pipeline}. In addition to contributing this novel method, we created 3 different datasets collected from  a number of artists. These datasets are designed such that they expose SAR to different levels of challenges and variations in order to prove it is effectiveness and validity. We hope that people in the community will find our databsets useful to use and research upon. Our collaborating artists who demonstrated through their work a great sketching abilities, have come from diverse artistic and sketching backgrounds as some of them are graphical designers and others are interior designer with an average sketching experience of 7 years and a maximum of 10 years.

{\color{blue}We positively concluded that artistic strokes are unique to the artist who drew them as indicated by our classification model results. Our Computational model can distinguish the authorship of a sketch among 3 artists who have drawn the same character viewed from different orientations with 83\% accuracy on test data using 5-fold cross validation (chance is 33\%) which is significantly better than human performance with 42\% accuracy according to a user study we conducted and involved 1000 participants.  Moreover, in another experiment, several artists attempted to draw fraudulent sketches of an assigned set of various original sketches and SAR detected fraudulent sketches with 95\% accuracy (chance is 50\%). This is much higher than artists performance with 52\% accuracy according to our user study which involved 25 artists. When classifying the authorship on the later dataset among all the different artists, the authorship classification accuracy recorded is 50\% (chance is 14\%) which gives a clear sign that although the sketches look very similar our method proved there are distinguishable features between sketches among different artists. More details and further experimental results are presented in this work.

During our creation of SAR, we experimented different techniques with the aim to come up with the best computational model that would give us the highest classification accuracy across different datasets. Towards the end of this paper, we share a number experimental results regarding different algorithmic variations we adopted and tested so that the community can use our findings and take them into consideration when conducting a research in the same field. We experimented how using digitization techniques for strokes extraction can affect the originality of the sketch. We have also tested our proposed segmentation algorithm against random and manual segmentation methods to prove its value. Moreover, we experiment the effect of the content in terms of the dataset size and similarity(content dependent)and diversity (Content independent) of the sketches in the datasets on the performance and robustness of SAR. Finally, we found that the silhouettes of a sketch convey lots of information regarding the authorship of a sketch when compared to internal strokes

Towards the end of this paper we demonstrate a couple of applications which are built using the best performing computational model. The first one is an artistic fraud detection application that is based on one of the datasets we are providing. The second application is a training program that allows artists, designers and animators to test their affinity to any given artist, whose works have been incorporated into the machine learning part of the program, it is also designed based on one of our datasets.} {\color{red} [The part in blue is not an introduction! This reads like a part of the discussions and a summary of the methodology. An intro focuses on describing the problem statement at hand, why it is important, motivating it with applications, etc. At the very end of the intro, you can give a sneak peak at the actual method used for solving the problem at hand. \textbf{Please rewrite}]}

