In this section, we survey previous work in the literature that is most related to our SAR approach. We group this body of work into four main categories.

\vspace{-2mm}
\subsection{Shape Matching} \label{subsec: shapematching}
\vspace{-2mm}
Determining stroke authorship seems akin to shape matching, but it is quite different. Numerous methods have been developed for shape matching and classification in the past ~\cite{Wolfson90:0,mokhtarian1992theory,conf/cvpr/CohenH98,Latecki:2000:SSM:354167.354193,belongie2001matching,jin2003image,berg2005shape,Mori2005}, as well as, recently \cite{Michel:2011:SID:1994006.1994152,ion-cviu-11}. Shape matching searches for similar shapes between two images, where one is usually considered the query image. Our system differs from shape matching in two main aspects. First, shape matching focuses on global information of contours such as zero crossings of curvature, while SAR segments the contour and studies detailed local features from each curve segment e.g. the eccentricity of a conic fit. Second, shape matching usually works on small images with a hundred or less pixels for computational reasons, where the images are usually classified by their contents, e.g., a set of different mice silhouettes from different authors may belong to one class of object. The proposed SAR technique can be employed efficiently on images of various sizes, especially images with a large number of pixels. Each class of shapes represents one specific object. For example, SAR can predict who drew a particular sketch of a 'flower' even though all its training was performed on non-'flower' images. This is not possible with shape matching. Authorship of a figure is represented by its design style and \emph{not} the figure itself. 

%removed resources: \cite{Bai:2008:DRC:1361741.1361848} \cite{latecki2000shape}
\vspace{-2mm}
\subsection{Sketch and Artistic Style Analysis}\label{subsec: artisticanalysis}
\vspace{-2mm}
Berger et. al. \shortcite{Berger:2013:SAP:2461912.2461964} provide a data driven approach to analyze style and abstraction in portrait sketches. This technique is also used for portrait sketch synthesis. While their focus is to mimic and synthesize a particular sketching style of a particular class of sketches (human portraits), our focus is to discriminate and classify sketches of any type based on their authorship (i.e. artistic style). In their study, the authors analyze sketches at the level of strokes and shapes. However, they handle stroke analysis differently, since they focus on global stroke features, while SAR focuses on local stroke features that are necessary in discriminating authorship of sketches that look quite similar as is the case in sketch fraud detection. Moreover, digitally collecting portrait sketches using the Wacom pen enables them to build a library of strokes. This explicit sketch information is not accessible in general and can be considered in some cases to be invasive, since artists should be given the freedom to draw with whichever medium they prefer, including pen, pencil or digitally. In our experiments, we allowed artists to express their styles freely giving us the chance to communicate with them remotely. Unlike previous work, we allowed artists to erase or redraw parts (or the entirety) of their sketches. Interestingly, we use the dataset of synthesized and original portrait sketches generated in \cite{Berger:2013:SAP:2461912.2461964} to demonstrate how SAR can be used to automatically and quantitatively evaluate sketch synthesis results. SAR results are on par with those reported by the authors after an extensive online study with human subjects.

Limpaecher et. al. \cite{Limpaecher:2013:RDA:2461912.2462016} designed an iPhone game to collect and analyze 13,000 drawings of faces. Unlike our work, they do not study the problem of authorship as they focus on auto-correction of strokes for novice artists. Lu et. al. ~\cite{Lu:2012:HES} mimicked a particular artistic style by matching using filtered velocities and shape context. Concurrently, work in ~\cite{Kalogerakis:2012:mlhatching} synthesized new drawings using hatching styles of sketching that are learned by example, while Freeman et. al. \shortcite{Freeman03learningstyle} provided an example-based method to modify line drawings into different artistic styles. Also, Cole et. al. \shortcite{Cole:2008:PDL:1360612.1360687} studied where artists draw lines in sketches of objects such as tools, automobile parts and bones. They concluded that artists usually tend to draw similar lines and that they are consistent in where they draw them. In our experiments, we use the dataset of this work to show interesting new results regarding the uniqueness of sketch style despite strict sketching constraints imposed on the artist.

Finally, there exist a large body of work that focuses on generating artistically stylized rendering using 2D input images or videos of non-photorealistic rendering (NPR). We refer the reader to the survey of ~\cite{Kyprianidis:2013:TAS} for more details. Although this work targets the analysis of sketches and artistic styles of different sketches, it does not address the important problem of how authorship can be determined based on stroke cues manifesting themselves in sketches. %similar to our method in terms of analyzing similar sketches and conclude authorship of those sketches based on a classification model.

\vspace{-2mm}
\subsection{Sketch Recognition and Retrieval}
\vspace{-2mm}
Eitz et. al. \shortcite{eitz2012hdhso} developed an automated data-driven method to explore a large collection of hand-drawn sketches using drawings collected by many non-experts. Their primary goal was to represent sketch content to perform object recognition from a sketch and \emph{not} sketch style. As such, they does not address the problem of authorship recognition. Concurrently, Sun et. al. \cite{Sun:2012:SAH:2393347.2396429} proposed a system that provides real-time recognition and retrieval of semantically meaningful attributes of hand-drawn sketches. Their work has the advantage that it is not limited to pre-defined object classes and it uses a web-scale clipart image collection for its knowledge base. Other sketch retrieval methods are based only on geometric similarity between sketches ~\cite{Chalechale05sketch-basedimage,Shrivastava:2011:DVS:2024156.2024188,5674030}. Those methods do not learn by example which makes sketch retrieving an efficient nearest-neighbor problem but they lack the semantic understanding of sketches. Unlike our work, the field of sketch recognition and retrieval is based on classifying sketch content among a discrete number of semantic categories using some knowledge base or geometrical analysis. They do not provide comparisons among similar sketches or across artistic styles and authors. 

\vspace{-2mm}
\subsection{Forensic Handwriting Analysis}
\vspace{-2mm}
Forensic handwriting analysis is a well studied problem [Srihari and Leedham 2003] as are signature analysis, and more precisely, off-line feature based signature verification [Impedovo and Pirlo 2008], [Kovari and Charaf 2013] and [Rivard et al. 2013]. Handwriting analysis tools tend to be very specific to the domain, e.g., letter height, pitch, baselines, crossings etc. They do not generalize well to sketch analysis [Srihari and Shi 2004]. Their focus is on the individuality of letter and punctuation formation, flow and structure [Franke and Kppen 2001], whereas we investigate the uniqueness of hand-drawn strokes in a much more general context. 
%Forensic handwriting analysis is a well studied problem and entails a large body of previous work \cite{srisurvey03}. The aim here is to develop techniques that identify authorship by comparing hand-written samples of different people. Similarly, signature analysis and more precisely off-line feature based signature verification has also been studied for many years \cite{4603099}. More recent work can be found in \cite{Kovari2013247} and \cite{signature2013}. While all this work strives to prove the individuality of only handwriting and signatures \cite{srihari_2002_jfs} \cite{handwriting2001}, we investigate the uniqueness of hand-drawn strokes in general. As a result, features extracted in handwriting analysis are only applicable to that specific domain (e.g., height of a loop, pitch, baselines etc.) and not generalizable to sketch analysis \cite{1263248}. On the other hand, SAR provides a general framework for stroke representation using second-order curvature analysis and the Bag-of-Words (BoW) model, which to our best knowledge has not been explored in the area of handwriting analysis. %We believe that such representation was not utilized in the area of handwriting analysis and signature verification.
\vspace{-2mm}
\subsection{Graphical Based User Authentication }
\vspace{-2mm}
In the past few years, a number of user authentication techniques have been proposed to replace the conventional authentication methods i.e. username and passowrd. They are based on identifying user graphical input, such as doodles and simple sketches. 
PassDoodle is a graphical based light weight authentication mechanism which attempts to identify users by their handwritten designs (doodles). For authentication, the query doodle is first stretched on a regular grid. After that, the system performs a Gaussian convolution on the combined training doodles which creates a blurring effect used to distribute grid values around the training doodles. Authenticity is then determined with the distribution map \cite{varenhorst2004passdoodle}, \cite{Govindarajulu:2007:PMU:1322192.1322233}. The effectiveness of using passDoodles for user authentication is demonstrated in the study provided by Renaud \shortcite{Renaud20091}. Oka et. al. \shortcite{4761233}, on the other hand, provided a sketch based authentication method by extracting edge orientation pattern feature from a user sketch input and then apply a similarity measurement. Unlike SAR, these authentication methods do not build a classification model and they do not provide an analysis of different artistic styles. Moreover, they do not build an intermediate sketch representation of the user graphical input as SAR and thus user input should be kept simple with minimum number of strokes. SAR's discrimination, on the other hand, is based on determining the frequency in which an artist uses certain basic strokes in comparison with other artists which makes it independent of the complexity of the sketch input. Analysis of the shortcomings of various graphical based authentication methods can be found in \cite{gani2010new}.
%The features are determine based on the distribution of the doodle on a regular grid.






