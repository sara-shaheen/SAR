Are simple strokes unique to the artist who draws them?  If they are, then to what extent can this uniqueness be used to identify their authorship or to classify their sketches? Moreover, to what extent would training and imposing sketching constraints on artists alter their styles? To answer these questions, we develop the Stroke Authorship Recognition (SAR) approach, which distinguishes 2D digitized sketches from different authors by analyzing inherent characteristics of sketch strokes. The SAR method represents a sketch as a histogram of universal stroke segments shared among most artists. In this paper, we show that this stroke representation can determine the authorship of 2D sketches. We conduct extensive classification experiments on various sketch datasets. Our results validate the effectiveness of SAR in distinguishing the unique authorship of artists even when certain restrictions are placed on their sketching style.  Using SAR as their core technique, a number of important applications are developed including the detection of fraudulent sketches, a training application that helps artists learn a particular style as well as monitor their training progress, and the first quantitative measure to evaluate the quality of automatic sketch synthesis tools.


%Different artists and designers exhibit characteristics that distinguish their sketches.  This observation leads us to ask two main questions: Are simple strokes unique to the artist who draws them?  If they are, then to what extent can this uniqueness be used to identify their authorship or to classify their sketches? In an attempt to answer these questions, we develop a new method called Stroke Authorship Recognition (SAR) that distinguishes 2D digitized sketches from different authors by analyzing inherent characteristics of sketch strokes.  SAR examines the characteristics of stroke segments and represents each sketch as a histogram of universal stroke segments, which are common among most artists. In this paper, we show that this stroke representation is discriminative of sketch authorship. To this end, we conduct extensive classification experiments on several datasets comprising sketches of local artists. SAR distinguishes sketch authorship even when certain restrictions on sketching style exist. As one of its applications, SAR is able to detect fraudulent sketches with an accuracy of 95\%, which is a substantial improvement on human performance as validated by an extensive online user study with more than 2000 participants. Moreover, we show that SAR enables two other sketch-related applications. One one hand,SAR provides artists in-training with immediate feedback on how close their sketching style is to a particular target style and how their style has evolved over time. On the other, SAR provides the first quantitative measure to objectively evaluate the quality of automatic sketch synthesis tools.


%of SAR (fraud detection of sketches), several artists are asked to meticulously draw fraudulent sketches of a set of original drawings. On this dataset, SAR detected fraudulent sketches with an accuracy of 95\%. Based on an extensive user study with more than 1000 \B{2000 now right?} participants, we show that our automatic results substantially improve upon human performance. Moreover, we show that SAR enables two important sketch-related applications. (1) SAR provides artists in-training with immediate feedback on how close their sketching style is to a particular target style and how their style has evolved with time. (2) SAR provides the first quantitative metric to evaluate the quality of automatic sketch synthesis tools.

 %which assists artists by providing them with feedback related to how close they get to a particular artistic style over training. The second application uses SAR in evaluating the quality of synthesized sketches in terms of how close they are to the originals sketched by artists.



%Shedding the light on other uses of such automatic analysis, in this paper we provide SAR is automatic feedback to artists on

%To motivate the sheer difficulty of this task, we compare our automatic results with human performance under the same learning circumstances in an extensive user study with more than 1000 participants.


% talk about the applications of SAR

% should say that: We assume that different authors use the same set of strokes (our dictionary). However, the frequencies of using strokes from this universal dictionary tend to be more similar for drawings of the same artist than for drawings of different artists.

% stress later the importance of these datasets on the ability to compare future methods for this interesting problem.

